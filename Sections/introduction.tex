\section[Ambiguous Business Aliases]{Ambiguous Business Aliases\protect\footnote{This section was written in collaboration with Jan Ehmüller~\cite{ehmueller}.}}
\label{sec:introduction}
The graph of Germany's corporate landscape consists of businesses as nodes and their relations as edges. Although structured knowledge bases, such as Wikidata and DBpedia, provide extensive information, they miss information about the relations. These are described in unstructured texts like Wikipedia or newspaper articles.

We have developed an approach to extract a relation between two businesses if they are mentioned in the same sentence. Then both of these mentions need to be found and linked to the entities representing these businesses. Finding these mentions is called \textit{Named Entity Recognition} (NER). The next step is to link them to the correct entity, which is called \textit{Entity Linking} (EL). The eventual extraction the relation between these two businesses from the sentence is called \textit{Relation Extraction} (RE). Our approach combines NER and EL into a single process and reduces it into a classification problem. The combination of NER, EL and RE make up the Information Extraction component of our project.

This thesis focuses on the development of a reliable implementation for combined NER and EL. Since there is no one-to-one relationship between businesses and their aliases, this is a complex task. On the one hand, a business may have multiple aliases. For example, "Deutsche Bahn AG" is commonly abbreviated with "Deutsche Bahn" or simply "DB". On the other hand, aliases are ambiguous, which means that the same alias may refer to different businesses in different contexts. This holds true especially for abbreviations. For example, "DB" may also refer to "Deutsche Bank AG". We therefore extract expressive features that allow us to disambiguate such business aliases and link them to their actually meant entity.

Sec.~\ref{sec:related_work} covers related work in the field of both NER and EL. Sec.~\ref{sec:ner_el} explains how our combined NER and EL approach works, before Sec.~\ref{sec:features} characterizes which features we use to perform EL. Sec.~\ref{sec:preprocessing} then describes our data sources and how they are used for the feature extraction. After that, Sec.~\ref{sec:evaluation} discusses the quality of the features and the scalability of the implementation. Finally, Sec.~\ref{sec:conclusion} will summarize the results and what steps may improve the project in the future.
