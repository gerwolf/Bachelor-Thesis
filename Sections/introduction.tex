\section[Ambiguous Business Aliases]{Ambiguous Business Aliases\protect\footnote{This section was written in collaboration with Jan Ehmüller~\cite{ehmueller} and Alec Schneider~\cite{schneider}.}}
\label{sec:introduction}

This thesis is published in the context of a project with the goal to create a graph of Germany's corporate landscape. Businesses represent the graph's nodes and the relationships between businesses represent the edges. Information about businesses are extracted from structured knowledge bases such as Wikidata and DBpedia. They describe the businesses themselves but contain incomplete information about the relationships between them. Unstructured texts like Wikipedia or newspaper articles describe these relationships. % correction: passive -> active

For our approach to extract a relationship between two businesses they must first be mentioned in the same sentence. Then both of these mentions need to be found and linked to the entities representing these businesses. Finding these mentions is called \textit{Named Entity Recognition} (NER). They must then be linked to the correct entities in the knowledge base. This step is called \textit{Entity Linking} (EL). The last step is to extract the relationship between the two businesses from the sentence, which is called \textit{Relation Extraction} (RE).

Our approach combines NER and EL into a single process and transforms it into a classification problem. This work describes the creation of our knowledge base and the extraction of features used to classify mentions. Ehmüller~\cite{ehmueller} evaluates the quality of the features and different classification models. Schneider \cite{schneider} describes and evaluates different Relation Extraction methods. % correction: step -> process

Sec.~\ref{sec:related_work} shows which projects aimed on similar goals and how they influenced this project. Sec.~\ref{sec:ner_el} explains how our combined NER and EL approach works, before Sec.~\ref{sec:features} characterizes which features we use to perform the EL. Sec.~\ref{sec:preprocessing} then describes our data sources and how they are used for the feature extraction. Sec.~\ref{sec:evaluation} evaluates the quality of the features and the scalability of the implementation. At last, Sec.~\ref{sec:conclusion} will summarize the results and what steps may improve the project in the future. % correction: joint -> combined



\iffalse
We want to build a graph that represents the network of mainly German businesses. Structured knowledge bases, such as Wikidata, DBpedia and Implisense, provide extensive information about businesses for themselves. But they miss information about the relationships between them. Those relationships are described in German newspaper articles. Unlike the structured knowledge bases these articles were written for humans and not for machines. We therefore need a way to extract the relevant information from these articles.

The challenge of extracting relationships between businesses from natural language text consists of three subtasks:
\begin{itemize}
\item Firstly, we have to find mentions of businesses in the text. This is called named entity recognition (NER).
\item Secondly, we have to link these mentions to known businesses so that we know which business they denote. This is called named entity linking (EL).
\item And thirdly, we have to find out which relations between the mentions of known businesses the text represents.
\end{itemize}
Jan Ehmüller, Alec Schneider and I have developed a system that solves the first two tasks. I will present our method in the following. Alec Schneider has found a solution for the relation extraction task, which he describes in his bachelor thesis~\cite{Alec}.

We have implemented the named entity recognition by collecting aliases (names) of German businesses and organizations that we recognize in natural language text via string matching. The development of a reliable implementation for the named entity linking was much more complicated, since there is no one-to-one relationship between organizations and their aliases. On the one hand, an organization may have multiple aliases. For example, the German automobile manufacturer "Bayerische Motoren Werke" is often abbreviated with "BMW". On the other hand, aliases are ambiguous, which means that the same alias may mean different organizations in different contexts. This holds true especially for abbreviations. For example, "BMW" may also refer to the German rap group "Berlins Most Wanted".
Several features allow us to disambiguate such aliases and link them to their actually meant organization. These features are statistical values that describe how an alias occurs in natural language text.

In Sec.~\ref{sec:related_work} I will describe which projects aimed on similar goals and how they influenced this project. Sec.~\ref{sec:pipeline} describes our data sources and how they are used for the feature extraction before Sec.~\ref{sec:features} characterizes which features we use to perform named entity linking. Sec.~\ref{sec:el} then explains how the named entity linking works. Sec.~\ref{sec:evaluation} will evaluate the quality of our named entity recognition and named entity linking. At last, Sec.~\ref{sec:conclusion} will summarize the results and what steps may improve the project in the future.
\fi
