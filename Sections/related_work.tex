\section{Related Work}
\label{sec:related_work}
Various research teams have worked on projects, which aimed on automatically linking aliases from natural language text to German businesses as well. The following outline gives an overview about what they already have attained:

\subsection{Alias generation to improve company recognition in text}
Alexander Immer has worked on the bachelor project previous to this project~\cite{immer}. Starting from legal names, he researched how aliases can be generated by abbreviating and pruning them. To disambiguate ambiguous aliases, he uses two vector spaces: One for the tf-idf values of words occurring in the neighborhood of recognized aliases and another one for words occurring in newspaper articles in general. He achieved an f-score of 0.81 for the named entity recognition on German newspaper articles and an f-score of 0.79 for named entity linking. Our approach also uses two different vector spaces, but they are constructed in another way. In contrast to Alexander's system, we do not generate aliases by ourselves, but we retrieve them from the German Wikipedia.

\subsection{The CohEEL project}
The CohEEL project attempts to perform named entity recognition and named entity linking on natural language texts for general purposes~\cite{coheel}. Like our system, it also uses the features link score, page score and context score that are described in Sec.~\ref{sec:features}. We go a step further and also include second order features that are retrieved from those features and support the classifier.

\subsection{Stanford CoreNLP}
The Stanford CoreNLP has a component for named entity recognition that detects mentions of organizations in natural language texts~\cite{stanford}. In contrast to our approach, it uses a neuronal network (?). Also it does not link the recognized mentions to specific organizations, it yields comparable results regarding the NER.