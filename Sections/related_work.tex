\section{Related Work}
\label{sec:related_work}
Various research teams have worked on projects, which aimed at automatically linking aliases in natural language text to German businesses as well. The following outline gives an overview about what they already have attained:

Immer~\cite{immer} has worked on the project previous to this project. Starting from legal names, he researched how it is possible to generate aliases by abbreviating and pruning them. To disambiguate ambiguous aliases, he used two vector spaces: One for the tf-idf values of words occurring in the neighborhood of recognized aliases and another one for words occurring in newspaper articles in general. Our approach also uses tf-idf vector spaces. In contrast to Immer's system, we do not generate aliases, but we retrieve them from the German Wikipedia.

The CohEEL project~\cite{coheel} performs NER and EL on natural language texts for more general purposes than business entity linking. It also implements the features that our system uses. In contrast to this project, we perform NER and EL as one combined process.

Another peculiarity of our approach is the distributed execution through cluster computing. The next sections describes we find business aliases and their corresponding entities.

%The Stanford CoreNLP has a component for NER that detects mentions of organizations in natural language texts~\cite{stanford}. In contrast to our approach, it uses a neuronal network (?). Also it does not link the recognized mentions to specific organizations, it yields comparable results regarding the NER.