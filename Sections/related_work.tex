\section{Related Work}
\label{sec:related_work}
Various research teams have worked on projects, which aimed on automatically linking aliases in natural language text to German businesses as well. The following outline gives an overview about what they already have attained:

Immer~\cite{immer} has worked on the project previous to this project. Starting from legal names, he researched how aliases can be generated by abbreviating and pruning them. To disambiguate ambiguous aliases, he used two vector spaces: One for the tf-idf values of words occurring in the neighborhood of recognized aliases and another one for words occurring in newspaper articles in general. Our approach also uses two vector spaces, but they are constructed differently. In contrast to Immer's system, we do not generate aliases, but we retrieve them from the German Wikipedia.

The CohEEL project~\cite{coheel} attempts to perform NER and named EL on natural language texts for general purposes. It also implements some the features that our system uses. We go a step further and also include additional second order features that support our classifier.

In view of the above, the next section describes the working process of our system in detail.

%The Stanford CoreNLP has a component for NER that detects mentions of organizations in natural language texts~\cite{stanford}. In contrast to our approach, it uses a neuronal network (?). Also it does not link the recognized mentions to specific organizations, it yields comparable results regarding the NER.