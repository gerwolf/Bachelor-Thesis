\section{The German Corporate Graph Project}
\label{sec_project}

\subsection{Why a German Corporate Graph?}

When it comes to economic decisions, uncertainty is a critical issue. Following the approach of the rational choice theory, every market player is constantly trying to maximize his utility and to minimize his effort.

Uncertainty can be described as a lack of information of how a market - or here in the full German economic system - is constituted and about the future behavior of the market players. Presuming that every market player is acting on a rational basis, every information regarding his situation, resources, plans, and relations makes the results of his decisions more predictable. In this manner, we can state: The more relevant information a market player gathers about other players in the market or economy, the better the foundation of his decisions are. The broad range of that kind of information can lead to a significant competitive advantage. So it should be in a rational player's interest to collect as much relevant information as possible.
\medskip

In a connected economy, a lot of those uncertainties lay in the relations between corporations\footnote{We define as \emph{corporation} any juristic entity that takes part in the German economy. This includes especially businesses but also other entities like public corporations.}. This became evident in the so called \emph{Abgas-Skandal} or \emph{Dieselgate} of the Volkswagen AG in 2015, wherein a lot of external suppliers went into a spin, from a financial perspective \cite{stuttzeit}, \cite{automobilwoche}. This happened although most of the suppliers did not take part in the scandal itself. Since there are a lot of other examples like the \emph{Lehmann Brothers bankrupt} or any other kind of economic shock event, we can state that relations are a significant factor in the economic evaluation of corporations and their financial risks.
\medskip

Because there are millions of corporations in the German economy\footnote{ The \emph{Federal Bureau of Statistics} is noting 3,469,039 businesses in Germany in 2015 \cite{destatis1}. Following our definition of corporations, this number has to be seen as a lower bound for the total number of corporations in Germany.} and each corporation can potentially holding relations to hundreds or thousands of other corporations, the aim to collect and to overview all those relations becomes a complicated matter.
\medskip

This is where the \emph{German Corporate Graph Project} offers IT-based solutions. The project\grq s purpose is to explore and evaluate ways, methods, and obstacles on the journey from collecting information about relations to transform them into a graph representation on a big scale. Therefore the project participants developed a software pipeline. 
\newpage
This software pipeline covers the following steps and components:
\begin{itemize}
\item Normalization of attributes to a common shape
\item Automatic graph generation from structured data sources
\item Deduplication: Duplicate detection and data-fusion 
\item Information extraction: EL and NER on semi-structured and unstructured data sources
\item Curation interface: A web interface to control the whole pipeline, to aggregate pipeline statistics and to curate graph data by hand
\item Corporate Landscape Explorer: A web interface to explore the \emph{German Corporate Graph} visually   
\end{itemize}


\subsection{One project - seven contributions}

This thesis is published as a part of a bachelor\grq s project in 2016/2017 at Hasso-Plattner-Institute in Potsdam, Germany. The project\grq s objective was to build the \emph {German Corporate Graph}, like described above, for Germany\grq s corporate landscape. The project lasted ten months and was accompanied by Commerzbank AG, Germany. As a result, the project participants published several theses. 



See here a list of all published theses within the project\grq s context:

\begin{itemize}
\item Pabst explores \emph{Efficient Blocking Strategies on Business Data} \cite{pabst}.
\item Löper and Radscheit evaluate duplicate detection in their thesis \emph{Evaluation of Duplicate Detection in the Domain of German Businesses} \cite{loeperradscheit}.
\item Schneider\grq s thesis is entitled \emph{Evaluation of Business Relation Extraction Methods from Text} \cite{schneider}.
\item Janetzki investigates \emph{Feature Extraction for Business Entity Linking in Newspaper Articles}.
\item Ehmüller explores the \emph{Evaluation of Entity Linking Models on Business Data} \cite{ehmueller}.
\item \emph{Graph Analysis and Simplification on Business Graphs} is the title of Gruner\grq s thesis \cite{gruner}.
\item Strelow investigates the \emph{Distributed Business Relations in Apache Cassandra} \cite{strelow}.

\end{itemize}