\section[The German Corporate Graph Project]{The German Corporate Graph Project\protect\footnote{This section was written by Matthias Radscheit~\cite{loeperradscheit}}} %https://tex.stackexchange.com/a/153340



\subsection{Why a German Corporate Graph?}
When it comes to economic decisions, uncertainty is a critical issue. Following the rational choice theory approach, every market player is constantly trying to maximize his utility and minimize his effort.

Uncertainty can be described as a lack of information of how a market - or herein the full German economic system - is constituted and about the future behavior of the market players. Presuming that every market player is acting on a rational basis, all information regarding his situation, resources, plans, and relations makes the results of his decisions more predictable. In this manner, we can state: The more relevant information a market player gathers about other players in the market or economy, the better the foundation of his decisions is. The broad range of that kind of information can lead to a significant competitive advantage. So it should be in a rational player's interest to collect as much relevant information as possible.

In a connected economy, a lot of those uncertainties lie in the relations between corporations\footnote{We define as \emph{corporation} any juristic entity that takes part in the German economy. This includes especially businesses but also other entities like public corporations.}. This became evident in the so called \emph{Abgas-Skandal} or \emph{Dieselgate} of the Volkswagen AG in 2015, wherein a lot of external suppliers spin out of control, from a financial perspective \cite{automobilwoche,stuttzeit}. This happened although most of the suppliers did not take part in the scandal itself. Since there are a lot of other examples like the \emph{Lehmann Brothers bankruptcy} or any other economic shock event, we can state that relations are a significant factor in the economic evaluation of corporations and their financial risks.

Because there are millions of corporations in the German economy\footnote{The \emph{Federal Bureau of Statistics} notes 3,469,039 businesses in Germany in 2015 \cite{destatis1}. Following our definition of corporations, this number has to be seen as a lower bound for the total number of corporations in Germany.} and each corporation can potentially hold relations to hundreds or thousands of other corporations, collecting and to overseeing all those relations becomes a complicated matter.

The \emph{German Corporate Graph Project} is one approach to solve this problem. The project's purpose is to extract business entities from multiple structured knowledge bases (e.g. Wikidata and DBpedia), merge them, enrich them with relations extracted from unstructured documents and finally display the graph so that it can be visually explored.\par

The project consists of a pipeline, which starts with the import and normalization of structured knowledge bases. The next step is the Deduplication, which is the detection and fusion of occurrences of the same entity over multiple knowledge bases. These entities form a graph, whose nodes are businesses and whose edges are the relations between them. This graph is then enriched during the Information Extraction. In this step relations between entities are extracted from unstructured documents using Named Entity Recognition, Entity Linking and Relation Extraction. 

The results of all these steps can be viewed and curated in the so-called Curation Interface. This is a web-interface, which can be used to control the pipeline itself, view statistical data generated by other pipeline steps and to view and curate the entities and relations of the graph itself. The final graph can be visually explored by using the Corporate Landscape Explorer, which is also a web-interface.



\subsection{One project - seven contributions}
This thesis is published as a part of a bachelor's project in 2016/2017 at Hasso-Plattner-Institute in Potsdam, Germany. The project's objective was to build the \emph {German Corporate Graph}, like described above, for Germany's corporate landscape. The project lasted ten months and was accompanied by Commerzbank AG, Germany. As part of the process, the project participants published several theses. 

See here a list of all published theses within the project's context:

\begin{itemize}
\itemsep-0.1cm
\setlength{\itemindent}{-.1in}
\item Pabst explores \emph{Efficient Blocking Strategies on Business Data}~\cite{pabst}.
\item Löper and Radscheit evaluate duplicate detection in their thesis \emph{Evaluation of Duplicate Detection in the Domain of German Businesses}~\cite{loeperradscheit}.
\item Schneider's thesis is entitled \emph{Evaluation of Business Relation Extraction Methods from Text}~\cite{schneider}.
\item Janetzki investigates \emph{Feature Extraction for Business Entity Linking in Newspaper Articles}.
\item Ehmüller explores the \emph{Evaluation of Entity Linking Models on Business Data}~\cite{ehmueller}.
\item \emph{Graph Analysis and Simplification on Business Graphs} is the title of Gruner's thesis~\cite{gruner}.
\item Strelow investigates \emph{Distributed Business Relations in Apache Cassandra}~\cite{strelow}.
\end{itemize}