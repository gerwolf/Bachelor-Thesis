\section{Ambiguous Business Aliases}
We want to build a graph that represents the network of mainly German businesses. Structured knowledge bases, such as Wikidata, DBpedia and Implisense, provide extensive information about businesses as single entity. But they miss information about the relationships between them. Those relationships are described in German newspaper articles. Unlike the structured knowledge bases these articles were written for humans and not for machines. We therefore need a way to extract the relevant information from these articles.

The challenge of extracting relationships between businesses from natural language text consists of three subtasks: Firstly, we have to find mentions of businesses in the text. This is called named entity recognition (NER). Secondly, we have to link these mentions to known businesses so that we know which they mean. This is called named entity linking (NEL). And thirdly, we have to find out which relations between the mentions of known businesses the text represents.
Jan Ehmüller, Alec Schneider and me have developed a system that solves the first two tasks. I will present our method in the following. Alec Schneider has found a solution for the relation extraction task, which he describes in his bachelor thesis~\cite{Alec}.

We have implemented the named entity recognition by collecting aliases (names) of German businesses and organizations that we recognize in natural language text via string matching. The development of a reliable implementation for the named entity linking was much more complicated, since there is no one-to-one relationship between organizations and their aliases. On the one hand, an organization may have multiple aliases. For example, the German automobile manufacturer "`Bayerische Motoren Werke"' is often abbreviated with "`BMW"'. On the other hand, aliases are ambiguous, which means that the same alias may mean different organizations in different contexts. This holds true especially for abbreviations. For example, "`BMW"' may also refer to the German rap group "`Berlins Most Wanted"'.
Several features allow us to disambiguate such aliases and link them to their actually meant organization. These features are statistical values that describe how an alias occurs in natural language text.

In Sec.~\ref{sec:related_work} I will describe which projects aimed on similar goals and how they influenced this project. Sec.~\ref{sec:pipeline} describes our data sources and how they are used for the feature extraction before Sec.~\ref{sec:features} characterizes which features we use to perform named entity linking. Sec.~\ref{sec:nel} then explains how the named entity linking works. Sec.~\ref{sec:evaluation} will evaluate the quality of our named entity recognition and named entity linking. At last, Sec.~\ref{sec:conclusion} will summarize the results and what steps may improve the project in the future.