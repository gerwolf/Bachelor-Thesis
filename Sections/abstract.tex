\begin{abstract}{Abstract}
German newspaper articles contain a lot of recent information about business relations. Their automated retrieval allows to construct the \textit{German Corporate Graph} and keep it up to date. This can be done by means of NER and NEL to find the mentioned businesses. Since business aliases may be ambiguous, this is a complex problem. Its solution requires the extraction and comparison of expressive features for business entity linking.

This research work comprises a software system that finds and disambiguates references to organizations in newspaper articles by means of appropriate features. It uses the German Wikipedia that contains explicit link annotations to named entities and learns from it to recognize mentions of organizations. The extracted features are statistical measures, linguistic properties of the alias' context and second order features. The system then applies these features to newspaper articles without link annotations, which allows to find business aliases and their referenced entities.

The distributions of the features' values reveals that they strongly depend on whether a reference is valid or not. This means that the features have a high quality and are applicable by a classifier. Furthermore, the software is scalable in order to be also suitable for economical use on large amounts of data. 
\end{abstract}

\newpage
\begin{otherlanguage}{ngerman}
\begin{abstract}{Zusammenfassung}
Deutsche Zeitungsartikel enthalten eine Menge aktueller Informationen über Unternehmensbeziehungen. Ihre automatische Extraktion ermöglicht es, den \textit{Deutschen Unternehmensgraphen} zu konstruieren und auf dem neusten Stand zu halten. Dies kann mithilfe von NER und NEL erreicht werden, um die genannten Unternehmen zu finden. Da Unternehmensnamen mehrdeutig sein können, ist dies eine komplexe Aufgabe. Ihre Lösung erfordert die Extraktion und den Verleich von aussagekräftigen Merkmalen zur Unternehmenserkennung.

Diese Forschungsarbeit umfasst ein Softwaresystem, das Verweise auf Organisationen in Zeitungsartikeln mithilfe von angemessenen Merkmalen eindeutig ermittelt. Es benutzt die deutschsprachige Wikipedia, die explizite Linkannotationen auf benannte Entitäten enthält, und lernt daraus, Nennungen von Organisationen wiederzuerkennen. Die extrahierten Merkmale sind statistische Maße, linguistische Eigenschaften des Kontexts eines Alias' und Merkmale zweiter Art. Das System wendet diese Merkmale dann auf Zeitungsartikel ohne Annotationen an, was das Finden von Unternehmensnamen und ihren referenzierten Entitäten ermöglicht.

Die Verteilung der Merkmalswerte zeigt, dass diese stark davon abhängen, ob eine Referenz gültig ist oder nicht. Das heißt, dass die Merkmale eine hohe Qualität besitzen und von einem Klassifikator verwendet werden können. Darüber hinaus ist die Software skalierbar, um auch für große Datenmengen wirtschatlich einsetzbar zu sein.
\end{abstract}
\end{otherlanguage}