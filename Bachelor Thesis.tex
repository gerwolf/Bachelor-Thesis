\documentclass[
        a4paper,     % Format A4
        titlepage,   % mit Titelseite
        parskip      % mit Durchschuss
                     % (= Abstand zwischen Absätzen, statt Einrückung)
        ]{scrartcl} % KOMA-Script Grundklasse     texdoc scrguide

\usepackage[USenglish]{babel}
\usepackage[T1]{fontenc}          % Schriftkodierung mit Umlauten
\usepackage{textcomp,amsmath}     % Mathezeichen etc.
\usepackage{graphicx}             % Graphiken einbinden
\usepackage[utf8]{inputenc}				% direkte Eingabe von Umlauten & Co. (Vorsicht: Encoding im Editor muss auch UTF-8 sein!)

% bibtex
\usepackage{url}
\bibliographystyle{plaindin}      % BibTeX Styles nach Norm DIN 1505

\renewenvironment{abstract}
% https://tex.stackexchange.com/questions/151583/how-to-adjust-the-width-of-abstract
 {\small
  \begin{center}
  \bfseries \abstractname\vspace{-.6em}\vspace{0pt}
  \end{center}
  \list{}{
    \setlength{\leftmargin}{.6cm}%
    \setlength{\rightmargin}{\leftmargin}%
  }%
  \item\relax}
 {\endlist}

\titlehead{
\includegraphics{Graphics/hpi_logo_cmyk_wb_sl2}
} \subject{Bachelor Thesis}
\title{Feature Extraction for Business Entity Linking in Newspaper Articles
\\ \bigskip 
\large{Merkmalsextraktion zur Unternehmenserkennung in Zeitungsartikeln}}
\author{Jonathan Janetzki\\{\small{\url{jonathan.janetzki@student.hpi.de}}}}
\date{July 21, 2017}
\publishers{
Information Systems Group\\
~\\
\textbf{Supervisors}\\
Prof. Dr. Felix Naumann\\
Toni Grütze\\
Michael Loster}


\pagestyle{headings}    % Seitenstil mit Kapitelüberschriften in der Kopfzeile


\begin{document}
\maketitle

\newpage
\begin{abstract}{Abstract}
German newspaper articles contain a lot of recent information about business relations. Their automated retrieval allows to construct the \textit{German Corporate Graph} and keep it up to date. A premise for the relation extraction is NER and EL to identify the mentioned businesses. Since business aliases may be ambiguous, this is a complex problem. Its solution requires the extraction and comparison of significant features for business entity linking.

This thesis comprises a software system that finds and disambiguates references to organizations in newspaper articles using appropriate features. It uses the German Wikipedia that contains explicit link annotations to named entities and learns from it to recognize mentions of organizations. The extracted features are statistical measures, linguistic properties of the alias' context and second order features. The system then applies these features to newspaper articles without link annotations, which allows to find business aliases and their referenced entities.

The distributions of the features' values reveals that they strongly depend on whether a reference is valid or not. This means that the features have a high quality and are applicable by a classifier. Furthermore, the software is scalable in order to be also suitable for economical use on large amounts of data. 
\end{abstract}

\newpage
\begin{otherlanguage}{ngerman}
\begin{abstract}{Zusammenfassung}
Deutsche Zeitungsartikel enthalten eine Menge aktueller Informationen über Unternehmensbeziehungen. Ihre automatische Gewinnung ermöglicht es, den \textit{Deutschen Unternehmensgraphen} zu konstruieren und auf dem neusten Stand zu halten. Eine Voraussetzung zur Relationsextraktion ist NER und EL, um die genannten Unternehmen zu finden. Da Unternehmensnamen mehrdeutig sein können, ist dies eine komplexe Aufgabe. Ihre Lösung erfordert die Extraktion und den Verleich von aussagekräftigen Merkmalen zur Unternehmenserkennung.

Diese Forschungsarbeit umfasst ein Softwaresystem, das Verweise auf Organisationen in Zeitungsartikeln mithilfe von angemessenen Merkmalen eindeutig ermittelt. Es benutzt die deutschsprachige Wikipedia, die explizite Linkannotationen auf benannte Entitäten enthält, und lernt daraus, Nennungen von Organisationen wiederzuerkennen. Die extrahierten Merkmale sind statistische Maße, linguistische Eigenschaften des Kontexts eines Namens und Merkmale zweiter Art. Das System wendet diese Merkmale dann auf Zeitungsartikel ohne Annotationen an, was das Finden von Unternehmensnamen und ihren referenzierten Entitäten ermöglicht.

Die Verteilung der Merkmalswerte zeigt, dass diese stark davon abhängen, ob eine Referenz gültig ist oder nicht. Das heißt, dass die Merkmale eine hohe Qualität besitzen und von einem Klassifikator verwendet werden können. Darüber hinaus ist die Software skalierbar, um auch für große Datenmengen wirtschatlich einsetzbar zu sein.
\end{abstract}
\end{otherlanguage}
\newpage
{\small\tableofcontents}
\newpage
\section{Ambiguous Business Aliases}
e.g., Aldi (Aldi Nord / Aldi Süd)
\newpage
\section{Related Work}
\subsection{Alias generation to improve company recognition in text}
(Alexander Immer)
\subsection{The CohEEL project}
\subsection{Stanford CoreNLP}
\newpage
\section{Text Mining Pipeline}
\label{sec:pipeline}
This section describes how the raw data is preprocessed in order to train a classifier that performs NER and NEL on German newspaper articles.
~\\
Preprocesing on Wikipedia dump\\
Explanation of jobs and their runtimes\\
Tools: Stanford CoreNLP (Tokenizer), Apache Lucene (Stemmer)\\
Data structures: Trie (for alias recognition)\\

\subsection{Raw data}
The system uses Wikipedia and Wikidata as data sources to train the classifier that performs NEL. The following will describe both of them.

\subsubsection{Wikipedia}
As project aims on analyzing German newspaper articles, the German Wikipedia provides appropriate training data. It currently consists of 3.6 million articles [source]. The articles have 3 main components: - Natural language text makes up most of the data. - Links to other Wikipedia pages. - Structured content, such as infoboxes.\\
The system will train the classifier based on the text and the links. It discards the structured content, as most of it is part of DBpedia [source], which is already included in the company graph [source reference to other BA].\\
For faster processing, the system accesses this data via a dump of the Wikipedia [source link] that consists of XML and Wikimarkup. The whole dump has a size of 15.9 GiB.

\subsubsection{Wikidata}
The German Wikidata [source link] describes the same entities as the German Wikipedia. But in contrast to Wikipedia, it does not provide a textual description but an ontology for them. This includes the class of an entity, which can be a business or an organization [todo concrete identifier]. This information is used to train the classifier only with relevant data.\\
As Wikipedia, the system also accesses the German Wikidata via a dump that is stored as JSON. It has a size of 90.2 GiB.

\subsection{Overview}
The preprocessing of the raw data consists of multiple steps as depicted in Fig.~\ref{fig:job_dependencies}. The system
- filters the data sources for their relevant content and stores this into tables of an Apache Cassandra database. (Parsing)
- refines the links within the Wikipedia and counts the references between each alias and entity. (Link analysis)
- searches for aliases of organizations in the Wikipedia text. (Alias analysis) 
- analyzes the frequencies of words in the Wikipedia text. (Word analysis)
%- generates features to train the classifier. (Feature generation)
%- trains the classifier. (Classifier training)
%- performs NEL on newspaper articles. (NEL)

\begin{figure}[ht]
	\centering
  \includegraphics[width=0.7\textwidth]{Graphics/job_dependencies.png}
	\caption{Dependencies between the jobs of the text mining pipeline}
	\label{fig:job_dependencies}
\end{figure}


\subsection{Data preprocessing}
This subsection describes how the presented steps work in detail.

\subsubsection{Parsing}
For each article in Wikipedia, the parsing step converts the Wikimarkup to HTML by means of [source tool]. It then extracts the raw text and saves it into an entry in a Cassandra table. It also saves the contained links to other Wikipedia entities separately.

\subsubsection{Link analysis}
\subsubsection{Alias analysis}
\subsubsection{Word analysis}

\paragraph{Text parser}
\paragraph{Link cleaner}
\paragraph{Redirect resolver}
\paragraph{Link analysis}
\paragraph{Company link filter}
\paragraph{Link extender}
\paragraph{Trie builder}
\paragraph{Alias trie search}
\paragraph{Alias counter}
\paragraph{Document frequency counter}
\paragraph{Term frequency counter}
\paragraph{Cosine context comparator}

\subsection{Classifier training}
details following in the next section
\newpage
\section{Features for Business Entity Linking}
\subsection{CohEEL features}

\subsubsection{Link score}
Probability that alias is link\\
(Technically just a probability on Wikipedia as a sample)

\subsubsection{Page score}
Probability that an alias points to a specific page

\subsubsection{Link context score}
Bag of words retrieved form the text around a link (currently +- 20 words)\\
generate tf-idf vectors\\
compute cosine similarity between link vectors and page vectors

\subsection{Second order features}
\subsubsection{Rank}
\subsubsection{Difference to highest value}
$\Delta top$
\subsubsection{Difference to successive value}
$\Delta successor$

\subsection{Alternative contexts}
Helpful if Wikipedia context for organization is missing
\subsubsection{Sector context}
e.g., for automobile industry
\subsubsection{Location context}
e.g., for Munich
\subsubsection{Homepage context}
e.g., Bag of words on \url{https://www.sap.com/index.html}
\newpage
\section{Named Entity Linking in Newspaper Articles}
\label{sec:nel}
\subsection{Newspaper data sources}
Spiegel, Heise, Kompass, Gelbe Seiten, Bundesanzeiger
\subsection{Feature extraction}
\subsection{Feature recognition}
Random forest classifier

\subsection{Article based disambiguation}
Compare found named entities to others in the same article

\newpage
\section{Evaluation}
\label{sec:evaluation}
\subsection{Results}
Precision, Recall, F-Score\\
Wikipedia as ground truth dataset\\
Jan Ehmüller evaluates the features more thoroughly in his bachelor thesis~\cite{Jan}.

\subsection{System limitations}
\subsubsection{False positives}
\subsubsection{False negatives}
\newpage
\section{Conclusion and Outlook}
\subsection{Achievements}
\subsection{Possible improvements}
\newpage
\bibliography{References}
\newpage
\addcontentsline{toc}{section}{Glossary}
\section*{Glossary}
EL - Entity linking

\end{document}
